\begin{recipe}{Brezen-Kn"odel}{nach Alfons Schuhbeck}
  \inglist
  \ingredient{250 g Laugen Stangen\index{Laugengeb"ack}}
  \ingredient{250 ml Milch}
  \ingredient{2 Eier}
  \ingredient{1 Zwiebel}
  \ingredient{1 EL Öl}
  \ingredient{1 EL Petersilie}
  \ingredient{1 Prise Muskatnuss}
  \index{Kn"odel}

  \steps
  Von den Brezen Stangen das Salz entfernen, in \halb bis 1 cm große Würfel schneiden. Die
  Milch aufkochen, in die Eier rühren, mit Salz, Pfeffer und Muskatnuss würzen und mit den
  Brezen Würfeln vermischen, dabei aber nicht drücken.

  Die Zwiebel schälen, in kleine Würfel schneiden und in einer Pfanne im Öl bei milder Hitze
  glasig anschwitzen.  Mit der Petersilie in die Brezen Masse geben.

  Zwei Blätter starke Alufolie jeweils mit Klarsichtfolie belegen. Die Brezen-Knödel Masse
  darauf zu Rollen von etwa 5 cm Durchmesser formen. Die Enden der Alufolie erst etwas
  andrücken, dann drehen, so dass eine formschöne Rolle entsteht.

  Die Knödel Rollen in einem entsprechend großen Topf mit leicht siedendem Wasser etwa 30
  Minuten garen, aus dem Wasser heben, aus der Folie wickeln und heiß in Scheiben schneiden.
\end{recipe}
