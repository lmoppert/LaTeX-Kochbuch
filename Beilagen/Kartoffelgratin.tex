\begin{recipe}{Kartoffel-Zucchini-Gratin}{nach Dieter Müller}
  \inglist
  \ingredient{500 g festkochende Kartoffeln\index{Kartoffeln}}
  \ingredient{1 kleine Zucchini\index{Zucchini}}
  \ingredient{\halb Knoblauchzehe}
  \ingredient{200 g Sahne}
  \ingredient{100 ml Milch}
  \ingredient{1 Thymianzweig}
  \ingredient{1 Rosmarinzweig}
  \ingredient{20 g Butterwürfel}
  \index{Kartoffeln>Gratin}
  \index{Gem"use>Zucchini}
  \index{Gem"use>Kartoffeln}

  \steps
  Die Kartoffeln waschen, schälen und mit einem Gemüsehobel in gleichmäßige Scheiben 
  schneiden. Mit Küchenkrepp abtupfen. Die Zucchini waschen und ebenfalls in dünne
  Scheiben schneiden. 
  
  Die Knoblauchzehen schälen und eine feuerfeste Form damit ausreiben, dann mit Butter 
  auspinseln und Kartoffeln und Zucchini einschichten. Mit Salz und Pfeffer würzen.

  Sahne und Milch mit Rosmarin und Thymian einmal aufkochen und mit Salz und Pfeffer 
  würzen. Kräuterzweige entfernen und nach belieben noch mit frisch geriebenen Parmesan 
  mischen. 
  
  Die Milch vorsichtig über die Scheiben gießen, das Gratin mit Butterwürfeln belegen und 
  bei 200 \celsius im vorgeheizten Backofen etwa 30-40 Minuten backen.
\end{recipe}
