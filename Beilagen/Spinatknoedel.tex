\begin{recipe}{Spinatknödel}{aus Südtirol}
  \inglist
  \ingredient{300 g alte Brötchen}
  \ingredient{150 -- 200 ml Milch}
  \ingredient{800 g Spinat}
  \ingredient{1 Zwiebel}
  \ingredient{2 Knolauchzehen}
  \ingredient{50 g Butter}
  \ingredient{50 g Bergkäse}
  \ingredient{2 Eier}
  \ingredient{Muskatnuss}
  \ingredient{Parmesan}
  \index{Kn"odel}

  \steps

  Die Brötchen in kleine Würfel schneiden und in eine große Schüssel geben. Die
  Eier verquirlen und dazu geben. Das ganze kräftig mir Salz, Pfeffer und
  Muskatnuss würzen.

  Die Zwiebel und den Knoblauch schälen und fein würfeln. Die Butter in einer
  Pfanne auslassen und die Gemüsewürfel darin leicht anschwitzen. Die Milch in
  die Pfanne geben und ein paar Minuten köcheln lassen und dann auch zu den
  Brotwürfeln geben.

  Den Spinat waschen, schleudern und in einer Pfanne mit etwas Salz
  zusammenfallen lassen, bis alle Flüssigkeit weg ist. Anschließend in einem
  Sieb auskühlen lassen und dann auch in die Schüssel geben.

  Die Masse gut durchkneten und ein paar Minuten ruhen lassen. Anschließend zu
  kleinen Knödeln formen und im Salzwasser ca. 15 Min. leicht köcheln lassen.
  Herausnehmen und mit zerlassener Butter oder einer Pfifferlingsauce und
  geriebenem Parmesan servieren.

\end{recipe}
