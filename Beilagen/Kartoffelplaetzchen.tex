\begin{recipe}{Kartoffellplätzchen}{Kalifornische Beilage}
  \inglist
  \ingredient{400 g Pfifferlinge}
  \ingredient{1 Zwiebel}
  \ingredient{1 Knoblauchzehe}
  \ingredient{Kartoffeln}
  \ingredient{Speck}
  \ingredient{2 Eigelb}
  \ingredient{Butter}
  \ingredient{frische Kräuter}
  \ingredient{Speisestärke}
  \index{Kartoffeln>Kartoffelpl"atzchen}
  \index{Gem"use>Kartoffeln}

  \steps
  Die geschälten Kartoffeln gar kochen, abschütten und gut ausdämpfen lassen. Den Speck in
  feine Streifen schneiden und kross anbraten. Die Zwiebeln in feine Würfel schneiden und
  zum Speck geben und glasig andünsten.

  Die Kartoffeln zerdrücken, Speck Zwiebeln, etwas Speisestärke und die gehackten Kräuter
  (z.B. Petersilie, Salbei, Rosmarin oder Thymian) dazu geben, mit Salz und Pfeffer würzen
  und zu einem geschmeidigen Teig verarbieten. Den Teig mit Frischhaltefolie zu einer dicken
  Rolle fromen und zwei Stunden kalt stellen.

  Die Pfifferlinge säubern und in einer sehr heißen Pfanne mit Rapsöl anbraten. Nach drei
  Minuten etwas klein geschnittene Zwiebeln und gehackten Knoblauch zugeben und kurz
  mitbraten und mit Butter glasieren. Die Teigrolle in Scheiben schneiden und von beiden
  Seiten kurz anbraten.  Pilzsauce mit einem Stück butter glasieren und zu den
  Kartoffelnplätzchen servieren.

\end{recipe}
