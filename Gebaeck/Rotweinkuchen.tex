\begin{recipe}{Rotweinkuchen}{nach Alfons Schuhbeck}\label{Rotweinkuchen}

  \inglist
  \ingredient{280 g weiche Butter}
  \ingredient{240 g Zucker}
  \ingredient{1 EL Vanillezucker}
  \ingredient{6 Eier}
  \ingredient{280 g Mehl}
  \ingredient{1 TL Backpulver}
  \ingredient{1 TL Kardamon}
  \ingredient{1 TL Zimtpulver}
  \ingredient{\halb TL Piment}
  \ingredient{\halb TL Muskatnuss}
  \ingredient{\halb TL Nelkenpulver}
  \ingredient{100 g Zartbitterschokolade\index{Schokolade}}
  \ingredient{150 ml Rotwein\index{Rotwein}}

  \steps

  Den Backofen auf 175\celsius aufheizen. Eine Kuchenform mit Butter
  einfetten und mit Mehl bestäuben. Die Butter mit der Hälfte des Zuckers und
  dem Vanillezucker in einer Schüssel mit den Quirlen des Handrührgerätes
  cremig schlagen. Die Eier trennen und die Eigelb einzeln zur Buttermasse
  geben, unterrühren und die Masse schaumig schlagen.

  Das Mehl mit dem Backpulver mischen und in eine Schüssel sieben. Die
  Gewürze hinzufügen und untermischen. Die Schokolade grob hacken und
  ebenfalls mit untermengen.

  Die Eiweiße mit 1 Prise Salz zu einem vremigen Schnee schlagen, dabei den
  restlichen Zucker einrieseln lassen. Die Mehlmischung abwechselnd mit dem
  Wein und dem Eischnee unter die Buttermasse ziehen. Den Teig in die Form
  füllen und im Ofen auf der untersten Schiene 50--55 Minuten backen.

\end{recipe}
