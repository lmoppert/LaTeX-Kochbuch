\begin{recipe}{Gewürzlachs}{nach Dieter Müller}
  \inglist[Für die Beilage]
  \ingredient{4 Strauchtomaten}
  \ingredient{Salatspitzen}
  \ingredient{1 Artischocke}
  \ingredient{\halb TL Estragon}
  \ingredient{1 TL Olivenöl}
  \ingredient{1 Spritzer Essig}
  \ingredient{\halb Zitrone}
  \ingredient{4 Weißbrotscheiben}
  \ingredient{4 TL Olivenpaste}

  \inglist[Für den Lachs]
  \ingredient{500 g Lachsfilet}
  \ingredient{300 g Meersalz}
  \ingredient{200 g brauner Zucker}
  \ingredient{8 g schwarzer Pfeffer}
  \ingredient{2 Nelken}
  \ingredient{6 g Pimentkörner}
  \ingredient{5 g Thymian}
  \ingredient{10 g Fenchelsamen}
  \ingredient{2 Lorbeerblätter}
  \ingredient{8 g Wacholderbeeren}
  \ingredient{10 g Sternanis}
  \ingredient{8 g Korianderkörner}

  \inglist[Für die Sauce]
  \ingredient{1 EL Moutarde Violette*}
  \ingredient{2 EL Creme fraiche}


  \steps
  Die Artischocken zurecht schneiden, das Bodeninnere gut ausschaben und mit einer Zitrone
  abreiben.  Den Boden mit der Aufschnittmaschine in dünne Scheiben aufschneiden, diese
  gleich in kaltem Wasser mit Zitronensaft hell halten.  Danach in kochendem Wasser mit
  Salz, Zitronensaft und 2 EL Olivenöl nur einige Sekunden bißfest kochen und schnell in
  Eiswasser abkühlen.  Gut abtropfen und mit Olivenöl, weißem Balsamico-Essig, Estragon,
  Salz und Pfeffer vermischen und bei Zimmertemperatur ca. \halb Stunde ziehen lassen.

  Die Strauchtomaten häuten, Haube abschneiden und aushöhlen. Mit Salatspitzen und
  Kräutern garnieren und mit wenig Artischockenmarinade nappieren. Die Weißbrotscheiben
  beidseitig braun backen, abkühlen und dann mit Olivenpüree bestreichen. Für die Sauce
  die Zutaten verrühren und mit Salz und Pfeffer abrunden.

  Alle Zutaten mixen, mit 2 EL davon das Lachsfilet beidseitig würzen, in Klarsichtfolie
  einschlagen und ca. 1 Tag im Kühlschrank marinieren lassen.  Danach kalt abspülen,
  trocken tupfen und den Gewürzlachs weiter verarbeiten.  Die Gewürzmischung kann im
  geschlossenen Glas lange aufbewahrt werden.

  Lachsfilet portionieren und je zwei Scheiben auf kalte Teller platzieren. Sauce schön
  auftragen, je längs einen Senfstreifen aufspritzen und mit einem Holzspieß ein Muster
  ziehen.  Artischocken auf die Lachsscheiben verteilen, Friseesalat, Tomate und Crostini
  auflegen und nach Wunsch mit Blüten servieren.

  \textit{* Moutarde Violette ist ein außergewöhnlich aromatischer Senf. Die Farbe bekommt er
  durch blaue Trauben. Er wird nach einem alten Rezept von französischen Mönchen
  hergestellt.}
\end{recipe}
