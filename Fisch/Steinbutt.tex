\begin{recipe}{Steinbutt mit Spinat und Meerrettichsauce}{frei nach Christian Rach}
  \inglist[Für die Sauce:]
  \ingredient{2 Schalotten}
  \ingredient{20 g Butter}
  \ingredient{100 g kalte Butter}
  \ingredient{150 ml Noilly Prat}
  \ingredient{150 ml Gemüsefond}
  \ingredient{150 ml Sahne}
  \ingredient{50 g Meerrettich-Frischkäse}
  \ingredient{1 EL Dijonsenf}
  \ingredient{1 Lorbeerblatt}

  \inglist[Für den Fisch:]
  \ingredient{40 g Butter}
  \ingredient{600 g Steibuttfilets\index{Fisch>Steinbutt}}
  \ingredient{1 EL Mehl}

  \inglist[Für den Spinat:]
  \ingredient{300 g Blattspinat}
  \ingredient{2 Knoblauchzehen}
  \ingredient{30 g Butter}

  \steps

  Statt Steinbutt kann man auch Seezunge, Kabeljau oder Wolfsbarsch verwenden.

  Für die Meerrettichsauce die Schalotten schälen, würfeln und in 20 g Butter
  farblos anschwitzen. Mit Noilly Prat ablöschen und vollständig einkochen,
  dann den Fond angießen. Das Lorbeerblatt hinzugeben und auf die hälfte
  reduzieren. Die Sahne zufügen und nochmal ein wenig reduzieren.
  
  Das ganze durch ein Sieb passieren und die Stücke gut ausdrücken. Die
  Flüssigkeit mit dem Meerrettich und dem Senf mischen und erhitzen. Die kalte
  Butter würfeln und mit einem Stabmixer einarbeiten und mit Salz und Pfeffer
  abschmecken.
  
  Den Spinat putzen und waschen. In Salzwasser kurz blachieren, kalt abschrecken
  und gut ausdrücken. Den Knoblauch schälen und fein würfeln. In 30 g Butter
  kurz anschwitzen und mit dem Spinat mischen. Mit Salz und Pfeffer würzen.
  
  Für den Fisch eine Pfanne mit 40 g Butter erhitzen. Den Fisch in Portionen
  teilen und leicht mit Mehl bestäuben. Mit Salz und Pfeffer würzen und innen
  leicht glasig anbraten. Den Spinat auf Teller verteilen, mit der Sauce
  begießen und die Fischfilets darauf drapieren.
  
 \end{recipe}
