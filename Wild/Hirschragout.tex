\begin{recipe}{Hirschragout}{aus der Steiermark}
  \label{Hirschragout}
  \inglist[Für die Marinade:]
  \ingredient{1 kg Hirschfleisch}
  \ingredient{100 ml Portwein}
  \ingredient{\viertel l Rotwein}
  \ingredient{\dreiviertel l Wildfond}
  \ingredient{3 mittelgroße Zwiebeln}
  \ingredient{1 Knoblauchzehe}
  \ingredient{2 Thymianzweige}
  \ingredient{2 Lorbeerblätter}
  \ingredient{4 Körner schwarzer Pfeffer}
  \ingredient{6 Wacholderbeeren}
  \ingredient{2 Zwiebeln}

  \steps

  Das Hirschfleisch in mundgerechte Stücke schneiden und die Zwiebeln fein
  hacken. Schmalz in einer Pfanne erhitzen und das Fleisch portionsweise scharf
  anbraten und anschließend zur Seite stellen. In der gleischen Pfanne die
  Zwiebeln goldgelb rösten und zunächst mit dem Portwein und dann mit dem
  Rotwein ablöschen und jeweils trocken kochen lassen. Dann den Fond hinzugeben
  und zugedeckt 1\halb -- 2 Stunden schmoren lassen.

  Ein paar Minuten vor Ende der Schmorzeit die Geüwrze hinzugeben und mit
  ziehen lassen. Am Ende mit Pfeffer, Salz, und Zucker abschmecken und mit
  Mondamin oder Butter abbinden. Nach belieben noch ein wenig Sahne an das
  Ragout machen und mit Apfelmuß oder Preiselbeeren abschmecken.

\end{recipe}
