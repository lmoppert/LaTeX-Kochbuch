\begin{recipe}{Coq Au Vin}{GU Kochbuch Nr. 1}
  \label{Coq Au Vin}
  \inglist
  \ingredient{2 Hähnchen\index{Gefl"ugel>H"ahnchen}}
  \ingredient{12 Schalotten}
  \ingredient{200 g Champignons\index{Gem"use>Champignons}\index{Pilze>Champignons}}
  \ingredient{5 EL Butter}
  \ingredient{4 Scheiben Speck}
  \ingredient{\halb l Rotwein}
  \ingredient{\viertel l Geflügelfond}
  \ingredient{3 Knoblauchzehen}
  \ingredient{1 Lorbeerblatt}
  \ingredient{\halb TL Thmian}
  \ingredient{1 EL Mehl}
  \ingredient{1 Bund Petersilie}

  \steps
  Die Hähnchen in Einzelteile zerlegen und von allen Seiten anbraten. Speck würfeln und
  zum Hähnchen geben.

  Die Hähnchen mit Cognac beträufeln. Den Wein und ausreichend Geflügelfond (siehe
  \pageref{Gefluegelfond}) zugießen, so dass alles bedeckt ist. Lorbeer, Thymian und
  gewürfelten Knoblauch zugeben und alles bei milder Hitze 30 Minuten ziehen lassen.

  Die Pilze vierteln und bei starker Hitze anbraten. Schalotten würfeln und kurz mit
  dünsten.

  Nach der Garzeit die Hähnchen aus der Sauce nehmen und warm stellen. Die Sauce auf
  \viertel einkochen lassen und mit etwas Butter binden. Champignons dazugeben und mit den
  Hähnchen und Weißbrot servieren.
\end{recipe}
