\begin{recipe}{Geschmorte Kaninchen-Keule}{nach Dieter Müller}
  \inglist
  \ingredient{4 Kaninchen-Keulen}
  \ingredient{500 g Mirepoix}
  \ingredient{1 EL Tomatenmark}
  \ingredient{300 ml Weißwein}
  \ingredient{100 ml Fond}
  \ingredient{1 Knoblauchzehe}
  \ingredient{1 Zweig Estragon}
  \ingredient{1 Zweig Thymian}
  \ingredient{1 Zweig Rosmarin}
  \ingredient{1 TL Pfefferkörner}
  \ingredient{4 Piment-Körner}
  \ingredient{2 Lorbeerblätter}
  \ingredient{1 Stern-Anis}
  \ingredient{1 EL gehackter Thymian}
  
  \steps
  Die Kaninchenkeulen im Bräter in Olivenöl anbraten, herausnehmen und zur Seite stellen.
  Das Mirepoix ebenfalls kräftig anbraten, die Hitze ein wenig reduzieren und das
  Tomatenmark zufügen. Wenn es zu duften beginnt, mit dem Weißwein ablöschen und
  glacieren. Dann den Fond, die Gewürze und das Fleisch zufügen und das Ganze im Ofen bei
  200 \celsius zugedeckt eine Stunde schmoren.

  Die Keulen warm stellen und den Schmorsud durch ein Sieb passieren, etwas einreduzieren
  und eventuell mit etwas Stärke binden. Mit Salz und Pfeffer abschmecken und den
  gehackten Thymian hineinmischen.
\end{recipe}
