\begin{recipe}{Kürbis-Suppe}{frei nach Schuhbeck und Lafer}
  \inglist
  \ingredient{500 g Muskat-Kürbis}
  \ingredient{2 Zwiebeln}
  \ingredient{1 Petersilienwurzel}
  \ingredient{750 ml Gemüsefond}
  \ingredient{150 ml Sahne}
  \ingredient{100 ml Kokosmilch}
  \ingredient{1 Knoblauchzehe}
  \ingredient{1 TL Currypulver}
  \ingredient{Chilipulver}
  \ingredient{1 Splitter Zimtrinde}
  \ingredient{\halb ausgekratzte Vanilleschote}
  \ingredient{40 g kalte Butter}
  \index{Gem"use>K"urbis}
  \index{Suppe>K"urbis}

  \steps
  Den Kürbis waschen, entkernen, schälen und in Würfel schneiden. Die Zwiebel und die
  Petersilienwurzel schälen und fein Würfeln. Etwas Rapsöl in einem Topf erhitzen und das
  Wurzelgemüse darin andünsten.

  Die Kürbiswürfel zugeben und das Ganze weitere 3 Minuten andünsten, dann mit dem Fond
  ablöschen, Sahne und Kokosmilch zugeben, aufkochen und bei milder Hitze 15 - 20 Minuten
  köcheln lassen

  Die Suppe mit dem Curry- und dem Chili-Pulver würzen, pürieren und durch ein Sieb
  passieren. Dann die Knoblauchzehe, Zimtrinde und die Vanilleschote in die Suppe geben,
  einige Minuten ziehen lassen und wieder entfernen.

  Die ausgelösten Kürbiskerne hacken, anrösten und mit etwas Kürbiskernöl auf die
  portionierte Suppe geben.
\end{recipe}
