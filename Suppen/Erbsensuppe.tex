\begin{recipe}{Erbsensuppe}{nach Mälzer / Witzigmann}
  \label{Erbsensuppe}
  \inglist[Zutaten]
  \inglist
  \ingredient{1 Bund Suppengrün}
  \ingredient{250 g Kartoffeln}
  \ingredient{500 g Schälerbsen}
  \ingredient{300 g durchwachsener Speck}
  \ingredient{1 Lorbeerblatt}
  \ingredient{3 Pimentkörner}
  \ingredient{2 l Brühe}
  \ingredient{2 EL Rotweinessig}
  \ingredient{Wiener Würstchen}
  \ingredient{2 Stängel Majoran}
  \ingredient{1 Bund Petersilie}
  
  \steps
  Suppengrün würfeln, Kartoffeln schälen und würfeln. Erbsen abspülen und
  abtropfen lassen. Speck würfeln und in einem großen Topf anbraten.
  Suppengrün, Kartoffeln, Lorbeer und das zerstoßene Piment zugeben und kurz
  mitbraten.

  Mit der Brühe ablöschen, dann die Erbsen dazu und das ganze mit Salz und
  Pfeffer würzen. Anschlißend gut 2 Stunden garen und dabei gelegentlich
  umrühren.

  Am Ende mit Essig abschmecken, die Würstchen in der Suppe erwärmen und mit
  dem Majoran und der Petersilie verfeinern.
\end{recipe}
