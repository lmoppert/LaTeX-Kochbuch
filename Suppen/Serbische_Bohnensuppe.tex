\begin{recipe}{Serbische Bohnensuppe}{aus der Konserve}
  \inglist
  \ingredient{1 Dose Bohnen\index{Gem"use>Bohnen}\index{Bohnen>Wei"se}}
  \ingredient{1 Dose Tomaten}
  \ingredient{\halb l Brühe}
  \ingredient{\halb TL Thymian}
  \ingredient{\halb TL Oregano}
  \ingredient{\halb TL Bohnenkraut}
  \ingredient{\halb TL Estragon}
  \ingredient{\halb TL Majoran}
  \ingredient{\halb TL Selleriesalz}
  \ingredient{1 TL Paprika edelsüß}
  \ingredient{1 Knoblauchzehe}
  \ingredient{1 Peperoni}
  \ingredient{100 g Speck}
  \ingredient{25 g Butter}
  \ingredient{250 ml Sahne}
  \ingredient{2 EL Speisestärke}
  \ingredient{2 - 3 Bockwürstchen}
  \index{Suppe>Serbische Bohnensuppe}

  \steps
  Die Brühe mit den Gewürzen aufkochen. Den Knoblauch schälen und mit den Peperoni fein hacken und in die Brühe geben. Die Tomaten klein schneiden und \halb Stunde mitkochen, dann die Bohnen ebenfalls hinzufügen.

  Den Speck würfeln und in der Butter auslassen. Das ganze zur Suppe geben. Die Sahne mit dem Mondamin verrühren und in die Suppe geben. Das ganze aufkochen und je nach Geschmack Bockwürstchen klein schneiden und in die Suppe geben.
\end{recipe}
