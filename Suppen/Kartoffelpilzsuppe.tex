\begin{recipe}{Kartoffelpilzsuppe}{frei nach Alfred Biolek}
  \inglist
  \ingredient{700 g Kartoffeln}
  \ingredient{2 Stangen Lauch}
  \ingredient{2 Frühlingszwiebeln}
  \ingredient{1 Kohlrabi}
  \ingredient{4 Karotten}
  \ingredient{100 g Butter}
  \ingredient{800 ml Wasser}
  \ingredient{1 Lorbeerblatt}
  \ingredient{1 Zweig Thymian}
  \ingredient{5 g getrocknete Steinpilze}
  \ingredient{1 Päckchen Sahne}
  \ingredient{1 Becher Creme Fra\^{\i}che}
  \index{Gem"use>Kartoffeln}

  \steps
  Die Kartoffeln schälen, waschen und in kleine Würfel schneiden. Lauch und
  Frühlingszwiebeln putzen und in schmale Ringe schneiden. Kohlrabi und
  Karotten schälen und in feine Streifen schneiden.

  Die Hälfte des Lauchs mit den Frühlingszwiebeln und den Kartoffeln in 1 EL
  Butter im Topf anschwitzen, mit dem Wasser ablöschen. Mit Salz, Pfeffer und
  Muskatnuss würzen. Thymian und Lorbeerblatt in einem Gewürzbeutel hinzugeben
  und ca. \halb Stunde köcheln lassen.

  Die Kohlrabi, Karotten und den Rest Lauch in einer Pfanne mit etwas
  Öl anbraten, dann mit ein wenig Wasser ablöschen und in 5 Minuten "`al
  dente"' dünsten. Die Steinpilze mit der Sahne im Topf aufkochen und pürieren.

  Am Ende der Garzeit den Gewürzbeutel aus der Suppe nehmen, alles pürieren.
  Creme Fra\^{\i}che und die Pilzsahne hinzugeben und nochmals pürieren. Am
  Ende das Gemüse unterrühren und evtl. mit Croutons servieren.
\end{recipe}
