\begin{recipe}{Ceasar Chicken Salad}{Klassiker aus Kanada}
  \inglist[Für das Dressing:]
  \ingredient{1  gekochtes Ei}
  \ingredient{Olivenöl}
  \ingredient{Anchovis}
  \ingredient{Parmesan}
  \ingredient{1  Knoblauchzehe}
  \ingredient{Zitronensaft}
  \ingredient{Senf}

  \inglist[Für den Salat:]
  \ingredient{250 g Hähnchenbrust\index{Gefl"ugel>H"ahnchenbrust}\index{H"ahnchenbrust}}
  \ingredient{500 g Rucola\index{Rucola}\index{Gem"use>Rucola}}
  \ingredient{250 g Champignons\index{Pilze>Champignons}}
  \ingredient{50 g Speck}
  \ingredient{100 g Parmesan}
  \ingredient{Croûtons}
  \index{Salat>Ceasar Chicken}

  \steps

  Das gekochte Ei in einer Schüssel zerdrücken und mit dem Olivenöl zu einem
  Brei pürieren. Die Anchovis ebenfalls mitpürieren, diese kann man aber auch
  weglassen und einfach mehr Salz verwenden. Den Parmesan (nicht zu wenig) fein
  reiben und unter das Dressing rühren.

  Den Zitronensaft, den Senf, die zerdrückte Knoblauchzehe beifügen und das
  ganze mit Salz und Pfeffer würzen. Falls das Dressing zu dickflüssig ist,
  kann man es mit etwas Wasser verdünnen.

  Die Hähnchenbruststreifen, Pilze und den Speck jeweils anbraten. Mit dem
  Rucola zu einem Salate vermengen. Den Salat mit Croûtons (am besten in
  Knoblauchbutter geröstet) und grob geriebenem Parmesan garnieren.

\end{recipe}
