\begin{recipe}{Lauchr"ollchen Gratin}{Klassiker}
  \label{Lauch-Gratin}
  \inglist[Lauchröllchen]
  \ingredient{4 Stangen Lauch}
  \ingredient{8 Scheiben Kochschinken}
  \ingredient{\halb l Gemüsebrühe}
  \ingredient{Senf}

  \inglist[Käse-Bechamel]
  \ingredient{80 g Butter}
  \ingredient{80 g Mehl}
  \ingredient{\halb l Gemüsebrühe}
  \ingredient{350 g Gouda}
  \ingredient{Paprikapulver}
  \ingredient{10 ml Weisswein}

  \steps

  Die Lauchstangen säubern und das äußerste Blatt abschneiden. Den Strunk
  entfernen und halbieren. Die Lauchrollen in der leicht gesalzenen Gemüsebrühe
  ca. 5 -- 10 Minuten gar kochen, dann abschütten und den Sud auffangen. Die
  Stangen etwas abkühlen lassen, während dessen die Schinkenscheiben mit Senf
  bestreichen und anschließend den Lauch darin einwickeln. Die Stangen in eine
  Auflaufform legen.

  Für die Bechamel die Butter in einem Topf schmelzen lassen und das Mehl
  einrühren. Mit ein wenig Weißwein ablöschen und anschließend nach und nach
  die Gemüsebrühe untererühren. Am Ende ca. \zweidrittel des geriebenen Käse einrühren
  und warten, bis dieser geschmolzen ist.  Die Sauce über die Röllchen geben
  und das Ganze mit dem restlichen Käse bestreuen.

  Den Gratin mit Paprika bestreuen und bei 200 -- 220 \celsius für ca. 15 -- 20
  Minuten überbacken. Mit Salzkartoffeln servieren.

\end{recipe}
