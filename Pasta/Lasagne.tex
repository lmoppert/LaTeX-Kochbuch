\begin{recipe}{Lasagne}{Heiß aus dem Ofen}
  \label{Lasagne}
  \inglist[Für das Ragout]
  \ingredient{500 g Gehacktes}
  \ingredient{1 Zwiebel}
  \ingredient{1 Knoblauchzehe}
  \ingredient{2 EL Tomatenmark}
  \ingredient{2 Dosen Tomaten}
  \ingredient{1 Möhre}
  \ingredient{200 ml Rotwein}
  \ingredient{1 EL Oregano}
  \ingredient{1 EL Paprikapulver}
  
  \inglist[Für die Bechamel]
  \ingredient{50 g Butter}
  \ingredient{50 g Mehl}
  \ingredient{\halb Milch}
  \ingredient{\viertel Geflügelfond}
  \ingredient{1 Prise Muskatnuss}
  
  \inglist[Weitere Zutaten]
  \ingredient{250 g Nudelplatten}
  \ingredient{200 g Gouda}
  \ingredient{50 g Parmesan}
  \ingredient{Basilikum}
  \index{Pasta>Lasagne}
  
  \steps
  Zwiebeln in feine Würfel scneiden, den Knoblauch in Scheiben schneiden und die Möhre
  fein Raspeln. Das Gehackte in etwas Olivenöl anbraten, bis es leicht anfängt zu bräunen, 
  dann das Tomatenmark dazu, kurz mitbraten und schießlich die Zwiebeln ebenfalls hinein
  und glasig dünsten.

  Das Hackfleisch mit dem Rotwein ablöschen und die Flüssigkeit vollständig verkochen
  lassen. Dann die Tomaten hinein geben und das ganze eine Stunde leicht köcheln lassen.
  10 Minuten vor Ende das Ragout mit Oregano, Paprikapulver, Salz und Pfeffer würzen.

  Für die Bechamel die Butter in einem Topf auslassen, das Mehl zugeben und glatt rühren.
  Die Milch nach und nach hinzu geben. Mit Salz, Pfeffer und Muskatnuss würzen.

  Eine Auflaufform fetten und die Saucen zusammen mit den Nudelplatten abwechselnd
  hineinschichten. Das ganze mit Käse bestreuen und bei 180 \celsius ca. 40 Minuten
  überbacken. Beim Servieren die Lasagne mit ein paar Basilikumblättern verzieren.

\end{recipe}
