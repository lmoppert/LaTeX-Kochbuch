\begin{recipe}{Frikadellen}{nach Lutz Moppert}
  \label{Frikadellen}
  \inglist
  \ingredient{500 g Hackfleisch\index{Fleisch>Gehacktes}\index{Hackfleisch}}
  \ingredient{1 Ei}
  \ingredient{1 altes Brötchen}
  \ingredient{100 ml Milch}
  \ingredient{100 g Speck}
  \ingredient{25 g Butter}
  \ingredient{2 Zwiebeln}
  \ingredient{2 Knoblauchzehen}
  \ingredient{1 TL Chili}
  \ingredient{2 TL scharfer Senf}
  \ingredient{3 Tropfen Orangenöl}
  \ingredient{1 TL Majoran}
  \ingredient{1 TL Oregano}
  \ingredient{50 g Paniermehl}
  \ingredient{2 Zweige Thymian}

  \steps
  Den Speck in Würfel schneiden, in einer Pfanne auslassen und leicht bräunen. Die Butter
  dazu geben und leicht braun werden lassen. Zwiebeln und Knoblauch würfeln und leicht
  mitdünsten. Die Milch zugeben, einige Minuten köcheln lassen.

  Die Rinde des Brötchens abschneiden und wegwefen. Den Rest in kleine Würfel schneiden
  und in eine große Schüssel geben. Die Milch darüber schütten und ca. 10 Minuten ziehen
  lassen.

  Die Milchmischung mit dem Ei, den Gewürzen, Salz und Pfeffer verrühren und das
  Hackfleisch gründlich unterkneten. Aus der Fleischmasse kleine Kugeln formen und diese
  in Paniermehl wenden.

  Die Frikadellen in der Pfanne mit etwas Butter anbraten und anschließend bei mittlerer
  Hitze zusammen mit den Thymianzweigen fertig garen.
\end{recipe}
