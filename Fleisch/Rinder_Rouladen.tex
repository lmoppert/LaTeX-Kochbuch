\begin{recipe}{Rinderrouladen}{nach Johann Lafer}
  \label{Rinderrouladen}
  \inglist
  \ingredient{4 Scheiben Rindfleisch}
  \ingredient{4 Zwiebeln}
  \ingredient{2 EL scharfer Senf}
  \ingredient{8 Scheiben durchwachsener Speck}
  \ingredient{1 Stange Lauch}
  \ingredient{2 Möhren}
  \ingredient{\viertel Selerieknolle}
  \ingredient{2 EL Tomatenmark}
  \ingredient{\halb l Rotwein}
  \ingredient{\halb l Fond}
  \ingredient{1 Lorbeerblatt}
  \ingredient{1 Zweig Thymian}
  \ingredient{2 Pimentkörner}
  \ingredient{4 schwarze Pfefferkörner}

  \steps

  Die Fleischscheiben etwas platt klopfen, auf einer Seite mit Senf bestreichen
  und mit den Speckscheiben belegen. \emph{Zwei} Zwiebeln in dünne Streifen
  schneiden und zusammen mit den Speck-Scheiben auf der Senfseite verteilen,
  dann mit Salz und Pfeffer würzen. Dann die Scheiben von der langen Seite her
  einklappen und aufrollen. Mit Küchengarn, Rouladenklammern oder Nadeln
  fixieren.

  Lauch, Möhren, Sellerie und die anderen Zwiebeln schälen und in grobe Stücke
  schneiden. Butterschmalz in einem Bräter erhitzen und das Fleisch darin
  kräftig anbraten. Das Fleisch raus nehmen und das Gemüse kräftig anrösten.
  Anschließend das Tomatenmark zugeben, kurz mitbraten und dann nach und nach
  mit dem Rotwein ablöschen. Am Ende sollte alle Flüssigkeit weg sein, dann mit
  Brühe auffüllen, das Fleisch, die Kräuter und die Gewürze zugeben und dann 2
  Stunden schmoren lassen.

  Die Rouladen aus dem Fond nehmen und warm stellen. Den Fond durch ein Sieb
  gießen und anschließend noch ein wenig einkochen. Mit Essig, Zucker, Salz und
  Pfeffer abschmecken und gegebenenfalls noch mit ein bischen Kartoffelstärke
  binden.
\end{recipe}
